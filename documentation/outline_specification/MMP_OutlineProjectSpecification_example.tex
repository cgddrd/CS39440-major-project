\documentclass[10pt,fleqn,twoside]{article}
\usepackage{makeidx}
\makeindex
\usepackage{palatino} %or {times} etc
\usepackage{plain} %bibliography style 
\usepackage{amsmath} %math fonts - just in case
\usepackage{amsfonts} %math fonts
\usepackage{amssymb} %math fonts
\usepackage{lastpage} %for footer page numbers
\usepackage{fancyhdr} %header and footer package
\usepackage{mmpv2} 
\usepackage{url}
\usepackage{enumitem}
\setlist{parsep=0pt,listparindent=\parindent}


% the following packages are used for citations - You only need to include one. 
%
% Use the cite package if you are using the numeric style (e.g. IEEEannot). 
% Use the natbib package if you are using the author-date style (e.g. authordate2annot). 
% Only use one of these and comment out the other one. 
\usepackage{cite}
%\usepackage{natbib}

\begin{document}

\name{Connor Goddard}
\userid{clg11}
\projecttitle{Estimation of Terrain Shape Using a Monocular Vision-based System}
\projecttitlememoir{Estimation of Terrain Shape Using a Monocular Vision-based System} %same as the project title or abridged version for page header
\reporttitle{Outline Project Specification}
\version{0.2}
\docstatus{Draft}
\modulecode{CS39440}
\degreeschemecode{G601}
\degreeschemename{MEng Software Engineering}
\supervisor{Dr. Fr\'ed\'eric Labrosse} % e.g. Neil Taylor
\supervisorid{ffl}
\wordcount{}

%optional - comment out next line to use current date for the document
%\documentdate{10th February 2014} 
\mmp

\setcounter{tocdepth}{3} %set required number of level in table of contents


%==============================================================================
\section{Project description}
%==============================================================================

For a robot requiring autonomous motion or travelling capabilities, the task of navigating through a (typically) unknown environment is one that plays a fundamental role in ensuring such a machine can successfully make safe, yet objective decisions on how best to traverse from a starting location to a target location over typically uneven and/or poorly modelled terrain. \\

As a problem-space, autonomous navigation can be de-composed into two key areas of focus; one of \textit{reactive} navigation and the other of \textit{deliberative} navigation. Reactive (or local) navigation is concerned with controlling the path of the robot through the immediate surrounding area, in particular focussing on the safe traversal around ``dangerous" terrain hazards including obstacles and precipices. Deliberative (or global) navigation is concerned with planning ``higher-level" paths through a larger area of the environment, and tends to focus on planning the most optimal path from one point to another, over necessarily the ``safest" path. In a typical autonomous navigation system, feedback from both reactive and deliberative sub-systems must be combined to choose a final path that balances safety and objectivity. \\

The ability to autonomously identify and avoid obstacles is a keen, and well-explored area of robotics research, with a variety of approaches now available adopting many types of sensor including sonar and laser scanning. An alternative approach that has enjoyed increasingly greater research focus over the past decade is that of vision-based obstacle avoidance, involving the analysis of images captured from one or more digital cameras in order to identify possible dangers currently situated within the robot's field of view. \\ 

Many solutions to vision-based obstacle avoidance have been suggested and published, with the majority proposing novel ways of combining ``standard" computer-vision algorithms in order to improve on previously published results, or to focus on specific types of obstacle avoidance (e.g. precipice or animal detection). \\

The project proposed for the CS39440 Major Project aims to combine recent advances in camera technology with appropriate computer vision technique in order to research and develop a system for estimating the shape of the terrain currently travelled by a wheeled robot. In addition, it is predicted that such a system could also provide an estimation of the speed and rotation/orientation of the robot as it follows a path along the surface (similar to Visual Odometry). \\

While similar vision-based systems have been produced previously, it does not appear that any of these systems have adopted appearance-based patch tracking in such a way as to prevent the need to provide a ``rigid" calibration model in exchange for an incremental, more flexible calibration procedure. 


%==============================================================================
\section{Proposed tasks}
%==============================================================================

As part of the project scope analysis, this section details a list of key tasks identified as necessary in order to achieve the aims of proposed project. To provide as realistic a prediction as possible at this time, any identified potential approaches (and associated issues), along with a prediction of current developer knowledge, were taken into account.

\begin{enumerate}
	\item \textbf{Analysis of Existing Approaches} \\\\ \indent Given the high variety of. In particular, attention will need to be focussed on the comparison between appearance-based, and feature-based tracking as a mechanism for determining the distance travelled between subsequent images. 
	\item \textbf{}
\end{enumerate}


%==============================================================================
\section{Project deliverables}
%==============================================================================

It is expected that the resulting software and documentation will be produced as a result of the work conducted throughout the major project.


\nocite{*} % include everything from the bibliography, irrespective of whether it has been referenced.

% the following line is included so that the bibliography is also shown in the table of contents. There is the possibility that this is added to the previous page for the bibliography. To address this, a newline is added so that it appears on the first page for the bibliography. 

\newpage
\addcontentsline{toc}{section}{Initial Annotated Bibliography} 

%
% example of including an annotated bibliography. The current style is an author date one. If you want to change, comment out the line and uncomment the subsequent line. You should also modify the packages included at the top (see the notes earlier in the file) and then trash your aux files and re-run. 
%\bibliographystyle{authordate2annot}
\bibliographystyle{IEEEannot}
\renewcommand{\refname}{Annotated Bibliography}  % if you put text into the final {} on this line, you will get an extra title, e.g. References. This isn't necessary for the outline project specification. 
\bibliography{mmp} % References file

\end{document}
