%\addcontentsline{toc}{chapter}{Development Process}
\chapter{Experiment Methods}
%
%This section should discuss the overall hypothesis being tested and justify the approach selected in the context of the research area.  Describe the experiment design that has been selected and how measurements and comparisons of results are to be made. 
%
%You should concentrate on the more important aspects of the method. Present an overview before going into detail. As well as describing the methods adopted, discuss other approaches that were considered. You might also discuss areas that you had to revise after some investigation. 
%
%You should also identify any support tools that you used. You should discuss your choice of implementation tools or simulation tools. For any code that you have written, you can talk about languages and related tools. For any simulation and analysis tools, identify the tools and how they are used on the project. 
%
%If your project includes some engineering (hardware, software, firmware, or a mixture) to support the experiments, include details in your report about your design and implementation. You should discuss with your supervisor whether it is better to include a different top-level section to describe any engineering work.  

This chapter aims to provide a discussion into the experiments implemented with respect to the investigation detailed in Chapter 1, providing an emphasis on the approaches that were adopted and the resulting actions of handling issues encountered.

The results subsequently collected from these experiments are presented in the Chapter 3.

\section{Collection of Appropriate Test Data}

Prior to beginning the implementation of any experiments, it was first necessary to identify and collect sets of sample images which would accurately represent the types of input expected upon the deployment of a completed system into a live scenario. 

\subsection{Image Data Requirements}

As discussed previously in Section \ref{assumptions}, for the purposes of simplifying the \textit{primary aims} presented in the working hypothesis, the assumption was drawn that at least in early stages of investigation, the system would make exclusive use of images captured from a single camera positioned to face in front of the robot that was also limited to moving in the same direction as the camera (i.e. forward-motion only).

Therefore when identifying appropriate test data for use in experiments, a the following characteristics were considered:

\begin{itemize}
	\item The images must provide a reasonable field of view of the current scene both above and below the horizon line (i.e it was important to capture objects within the scene located both far away and close to the camera)
	\item The images must show the act of forward translation through the current environment. This would be most obvious through the observation of a vertical displacement in the negative direction (i.e. downwards) visible in features located along the ground plane.
	\item The images must not show forward translation as horizontal movement across the image plane (i.e. no images captured from cameras looking out from the side of a robot moving in a forwards direction).
	\item The images must be of a size and quality reasonably expected of a standard ``point-and-shoot" consumer-grade camera.
	\item The images must have been originally captured in colour, using the ``default" colour space supported by the camera (typically this would be RGB for standard consumer-grade camera).
	 \item The images should demonstrate minimal change in rotation or pitch (i.e. should be taken across a flat surface), and should be 
\end{itemize}

In addition to these ``baseline" requirements, additional requirements were also defined with the intention for use within specific experiments focussing on the identification of particular aspects in motion behaviour. These secondary requirements were very much intended to be used on an ``as needed" basis and came with the possibility of the opposite statement to the ones detailed below being desired in certain situations:

\begin{itemize}
	\item The images should demonstrate minimal rotational motion (i.e. no examples of the robot turning to change direction)
	\item The images should capture terrain that is predominately flat and free of major obstacles both positive (e.g. rocks) and negative (e.g. pits).
\end{itemize}

Unfortunately due to time constraints, some the planned experiments could not be completed within the scope of the major project, and as a consequence of this, not all examples of images meeting every one these requirements were actually captured. However, it was still important to define these requirements before beginning any experimentation and given more time, would still prove to be valid.

\subsection{Existing Datasets}

At the beginning of the project, some time was initially devoted to searching for any existing datasets that could provide suitable imagery. One of the main advantages to using existing datasets, is that typically other previous projects have already had the opportunity to verify that the data is both accurate, and provides a sufficient level of variability that proves crucial in testing the robustness of the systems that make use them as part of their evaluation.

In the majority of cases, existing datasets also come pre-packaged with appropriately verified ground truth data, thereby preventing the need for projects that subsequently chose to use them having to produce their own ground truth results, consequently saving on both time and resources.

While a number of previously published datasets were found to be available \cite{ucl-dataset}, \cite{baker-dataset}, \cite{mpi-dataset}, unfortunately none were found to be suitable in relation to the requirements detailed in the previous sub-section.   

\subsection{Manual Datasets}

Following the lack of appropriate existing datasets with the field, it was deemed necessary that bespoke datasets would instead have to be created manually. While in the short term this meant an increased work load, it also presented the opportunity to capture datasets that would could specifically meet the needs of the investigation and its experiments.

\subsubsection{Camera Rig Setup}

In the pursuit of capturing image datasets, a wide range of approaches could have potentially been adopted. One initial idea considered requesting the use of one of several `Pioneer' robots owned by the Computer Science department at Aberystwyth University. By rigidly mounting a camera to the front of one of these small wheeled robots, and remotely instructing it to move forward by a set distance before manually triggering the camera, it would be achievable to capture a collection of images in which each demonstrated an equal level of displacement between that and the next image. 

As part of a particularly elaborate setup, it would have perhaps been feasible to provide an interface to the mounted camera (perhaps via USB or serial) before programming the Pioneer robot into automatically capturing images at set intervals, while following a pre-determined path through the environment (making use of the on-board sonar and odometry capabilities). 

While these approaches would certainly provide an ample solution, following a discussion with the project supervisor, it was deemed that, at such an early stage in an investigation that was already limited in time remaining, efforts would be better spent focussing on conducting actual research, as opposed to the collection of test data.

Nevertheless, a means of manually capturing image datasets was still required. As such, the decision was taken to adopt a much more `simplified' approach, that in exchange for greater manual involvement, could consequently be brought into service within a vastly shorter timeframe. 

Compared to the use of the robot, this approach was certainly less `sophisticated' in terms of its setup, consisting only of:

\begin{itemize}
	\item Two cardboard boxes (one rectangular and one angled);
	\item A consumer-grade ``point-and-shoot" camera (Panasonic Lumix DMC-FS18 (2011));
	\item A standard 30cm ruler;
	\item A standard spirit level;
\end{itemize}

However, as well as being quick to initially build, this setup would go on to prove to be a lot more portable and a great deal cheaper to modify and fix than one of the Pioneer robots. 


\subsubsection{Capturing Process}

The method behind the use of the `homemade' camera rig to capture the image datasets also proved simple in design. This was regarded as favourable, given that it solely relied on manual involvement where human error subsequently becomes a potentially big issue. 

Figure \ref{} below details the outline of the method used to capture an individual image dataset:

% ADD FIGURE SHOWING FLOW CHART FOR METHOD (STEP 1 - GO TO START POSITION)...

While the method for capturing images followed a ``fixed" procedure, it did allow for changes in the type of terrain captured, and the level of set displacement by which the camera was moved between subsequent images.

\subsubsection{Capture Locations}

A selection of locations were chosen for the purposes of capturing datasets. To ensure an appropriate level of variability, each dataset focussed on a different type of terrain environment that it was predicted a robot may face during live deployment.

Datasets were captured at five separate locations, split between indoor and outdoor sites to provide variation in lighting conditions in addition to terrain type:

\begin{enumerate}
	\item \textbf{Site 1:} Living room rug with brightly-coloured simple-shapes (Indoors) (Natural light entering from the right)
	\item \textbf{Site 2:} Brick-paved residential drive (Outside) (Natural light)
	\item \textbf{Site 3:} Slate-tiled footpath (Outside) (Natural light)
	\item \textbf{Site 4:} Heavily-patterned asian rug upon white-tiled floor (Inside) (Partial natural light from behind camera, partial artificial light (incandescent))
	\item \textbf{Site 5:} Slate chips (Outside) (Natural light)
\end{enumerate}

\subsubsection{Calibration of Physical Setup}

Although one of the aims of the investigation hypothesis was to avoid the need for calibration of the camera, it was decided that in the interests of good experiment practice, measurements regarding the physical setup of the camera rig should be taken in case of future need. 

Following the process outlined in the work of Campbell \textit{et al.} \cite{campbell}, the calibration would provide a future opportunity, should the need arise, to calculate the mapping between coordinates in the image plane and the ground plane.

The physical setup of the camera rig is shown in Figure \ref{}, where $h$ refers to the height of the camera from the ground, and $d$ refers to the distance between the front edge of the rig base and the position at which the principle ray from the camera lens intersects with the ground plane \cite{campbell}.

To establish the distance $d$, a target $T$ was positioned on the ground such that it lay within the centre of the camera's viewfinder (Figure \ref{}). The distance between the position of the target $T$ and the front of the rig was then manually measured in order to arrive at the final length. 

After also manually measuring the height $h$, it became possible to calculate the tilt $ \alpha $ of the camera using basic trigonometry:

\begin{equation}
	\tan(\alpha) = \frac{h}{d}
\end{equation}

\textbf{Note to the reader:}

The remaining steps of the calibration model outlined below did not form part of the main project investigation, however following the decision by the author to investigate them as part of improving their own understanding, they have been included in this report for completeness.

The next step was to define the relationship between the vertical offset of a point in the image $v$ and the associated vertical angle between that same point and the principle point along the ground plane $\beta$ \cite{campbell}:

\begin{equation}
	\tan(\beta) = (2v - V)\tan(\frac{VFOV}{2})
\end{equation}  

where $ V $ refers to the height of the image in pixels, and $VFOV$ represents the vertical field of view from the camera:

\begin{equation}
	VFOV = 2\tan^{-1}(\frac{d}{2f})
\end{equation}

where (in this case) $d$ represents the vertical dimension of the camera sensor (but can also represent the horizontal dimension if wishing to calculate the horizontal FOV), and $f$ represents the camera focal length \cite{bourke}.

Obtaining values for $d$ and $f$ involved consulting the manufacturer specifications document for the camera \cite{camera-spec}, in which the following information was identified:

\begin{itemize}
	\item \textbf{Sensor Type:} 1/2.33" CCD
	\item \textbf{Lens Aspect Ratio:} 4:3
	\item \textbf{Focal Length:} 5-20mm 
\end{itemize}

In calculating $d$, there was some confusion about how to establish the dimensions of the sensor from the published sensor type. It was assumed initially that the diagonal dimension of the camera would match that of dimension representing the sensor type (i.e. 1/2.33"). 

However following some further research, it came to light that this was not actually the case, and in fact the dimension listed as the sensor type was not representative of the true sensor size, but instead was one of many industry-standard type designations used to classify image sensors \cite{gum}. Subsequently, the vertical dimension of the camera sensor was obtained using an online guide providing dimensions for standard types of sensor \cite{bockaert}.

The final step in the calibration process published by Campbell \textit{et al.} \cite{campbell} involved recovering the distance $y$ from the observed point along the ground plane to the camera rig:

\begin{equation}
	y = \frac{h}{\tan{\alpha + \beta}}
\end{equation}

and the associated depth $z$:

\begin{equation}
	z = \frac{H\cos(\beta)}{\sin(\alpha + \beta)} 
\end{equation}

Through the use of this calibration model, the actual displacement of any point along the ground plane that is represented in the image plane, can be calculated by taking into account the depth between that point along the ground plane and the camera \cite{campbell}.


\subsection{Artificial Datasets}

While unfortunately never pursued in this project due to time limitations, an alternative approach to gathering robust testing data was to capture motion sequences set within a 3D virtual environment. 

A key advantage of creating artificial image datasets stems from the level of control that becomes available over properties including the choice of lighting, terrain textures and landscape geometry. It is this high degree of granularity that makes using computer-generated datasets a popular choice for many vision-based projects, perhaps especially within earlier stages of a project, where one is simply attempting to identify bugs in the system, rather than necessarily evaluating hypotheses. 

Other potentially very useful consequences of using computer-generated datasets include control over the amount of noise an image may contain (e.g. choice to include or remove specular reflections), and the high degree of speed at which new datasets adopting different condition configurations could be created almost ``on the fly" in order to test new ideas and approaches within an experiment. 

The creation of artificial datasets would certainly have provided an opportunity to test the robustness of the experiment methods against a much broader range of conditions than what was feasibly possible using only real-world imagery. However the use of datasets captured from real-world environments was still arguably the most crucial in terms of evaluating the performance of experiment methods, given that these would ultimately provide the closest representation to the input expected within a live deployment. 

% ADD EXAMPLES OF ARTIFICIAL DATASETS FROM PUBLISHED DATASETS

While there was no time available for the author to learn how to build 3D environments for use in artificial datasets, some simple 2D-shape image datasets were created as part of evaluating the correct functioning of the template-matching approach adopting geometric scaling (discussed further in Section \ref{tm-scaled}).

\section{Establishment of Ground Truth}

With respect to scientific investigation, the term ``\textit{ground truth}" refers to measurement data or results that are known to represent the \textit{absolute truth} of a particular condition or observation which can as such, subsequently be used to determine the accuracy of reported results from a system or method currently undergoing testing and evaluation.

In the specific context of computer vision, ground truth data will represent observations within image datasets that are known to actually exist in the real-world, and be accurate.

Due to the very nature of what ground truth data represents, generating such data can still typically only be accomplished through means of manual verification, where human observation continues to be deemed as the `state of art' in terms of providing the best possible accuracy available as to the \textit{fundamental} truth of a condition or observation \cite{kondermann}. A fundamental issue with this approach to the collection and verification of ground truth data, is that it tends towards being very costly in terms of both time and resources. Kondermann \cite{kondermann} states that this large requirement in effort to generate ground truth data in combination with its relatively limited applicability in terms of evaluation, consequently results in many vision-based projects failing to provide reference data that is either sufficiently comprehensive or accurate.

For particular types of reference data, some semi-automated methods do exist for gathering high-accuracy measurement results (e.g. extracting depth information about a 3D environment using laser scanning techniques) \cite{haltakov}. However, these approaches still require at least some form of manual verification in ensuring that this accuracy is at a level sufficient enough for it to be trusted as ground truth.

In terms of generating ground truth, one of the most challenging types of visual behaviour to document is optical flow \cite{kondermann}. This is because no sensor currently exists that is capable of detecting the effects of optical flow directly, and in most situations (except perhaps within very simple artificial datasets) attempting to perform manual annotation also proves to be an extremely difficult task to achieve with the kind of accuracy that is expected for use as ground truth data \cite{haltakov}. 

As this was a problem that had potential implications for this project, thoughts began to turn to ways of developing a utility application that could assist a human in manually generating ground truth data for the observed displacement of features across a series of images. 

\subsection{Ground Truth Utility Tool}

Following discussions with the project supervisor a simple, yet effective solution was devised for determining the vertical displacement between a series of images. 

For each image dataset, it was proposed that an additional pair of images would also be captured for the exclusive purpose of establishing ground truth. While obtained using the exact same method as the rest of the dataset, these two images would differ from the main collection by the fact that they would also contain a ruler positioned within the centre of each image. The choice to capture the ruler would be important, as the measurement increments it possessed would allow a user to clearly identify the same `feature' within both images (e.g. find the 10cm mark in both images).

The next stage involved the design and implementation of a small, GUI-based utility tool that would allow a user to perform the following actions:

\begin{enumerate}
	\item Load in the pair of dedicated ground truth source images;
	\item Using the mouse, click to select a number of `feature points' that the user deemed easily-identifiable within the first image (e.g. a particular measure mark along the length of the ruler);
	\item Cycle through each `feature point' identified in the first image, and click on the location corresponding to the new position of that same `feature point' within the second image;
	\item Export the calculated ground truth measurements for use in future experiments. 
\end{enumerate}

% ADD FIGURE SHOWING IDEA BEHIND CALIBRATION TOOL

Following this process, a collection of sparse measurements would be obtained detailing the calculated number of pixels between the Y-coordinate of a `feature point' located in the first image, and the Y-coordinate of the corresponding `feature point' subsequently located within the second image. As it would not be possible for a user to manually identify and select features at an individual pixel level, interpolation would be required to estimate the intermediate values in order to arrive at a recorded displacement for every row in the image.

Naturally, as with any means of human verification this approach was susceptible to human error, both in terms how accurately a user could position their mouse upon a feature, and also ensuring that they matched the correct features to each other. However, following the conclusions of a number papers within this topic area \cite{kondermann} \cite{haltakov} \cite{}, human verification was deemed to still provide the best possible level of accuracy for generating ground truth data relating to optical flow.

\textbf{Note to the reader:}

Within the context of the experiments that were conducted it was agreed by both the author and the project supervisor that in the interests of time, the expected behaviour predicted in the hypothesis (namely  ``\textit{greater vertical displacement is shown approaching the bottom of an image than towards the top}") would serve as sufficient `un-official' ground truth from which the performance of the methods implemented could be evaluated.

\section{Design of Experiment Setup}

Prior to beginning the implementation of specific experiments, it was first necessary to design and implement an underlying software infrastructure with the aim of maximising the \textit{efficiency}, \textit{flexibility} and \textit{repeatability} of the experiments conducted:

\begin{itemize}
	\item \textbf{Efficiency:} The testing infrastructure should execute each experiment in a way that makes efficient use of data and system resources (including but not limited to: imagery datasets, experiment parameters and host computer CPU and memory allowance).
	\item \textbf{Flexibility:} The testing infrastructure should provide support for conducting an experiment under multiple combinations of method parameters entered by the user prior to initiation.  
	\item \textbf{Repeatability:} The testing infrastructure should ensure that each run of an experiment is treated as new and conducted in complete isolation to any previous run. 
\end{itemize}

\subsection{Programming Languages}

For this project, it was primarily the selection of specific external libraries that determined the programming languages that would be adopted.

The original intention saw the entire project developed in C++, as this matched the `native' development language of the OpenCV library \cite{opencv} that was expected to provide a significant contribution to the implementation of the appearance-based similarity measures under investigation. While the OpenCV library did provide a comprehensive range of bindings for other programming languages including Python, Java and C\#, it was felt that choosing to develop in C++ would prove to be the most beneficial in terms of maximising available functionality (i.e. not all functions within OpenCV were available for alternative languages) and maximising the level of support available from both official documentation and the wider community. 

Following the completed implementation of the \textit{first experiment} (detailed in Section \ref{ex1}), this original intention was altered to instead change from C++ to Python as the primary development language. 

This decision was not taken lightly, with the author being very much aware that by choosing a language with which they were unfamiliar, there was a high probability of the project incurring delays resulting from both learning the actual language, while also having to most likely dedicate additional time to fixing ``common" bugs typically associated with being new to the language. In spite of this, it was decided that while it may potentially cost time in the short term, switching to Python would provide a number of key benefits in the long term. 

The first of these benefits was that in comparison to C++, the language provided significantly more functionality ``out of the box" due to it being situated at a ``higher" level programmatically. While reviewing the implementation of the first experiment, it was found that on a number of occasions, a function that had to written manually in C++ (e.g. standard deviation and variance of a numeric collection), was already available within the Python language itself, or one of its primary third-party libraries (e.g. numpy \cite{numpy} and scipy \cite{scipy}). While it was true that many of these functions did not take particularly long to implement in C++, there was a clear benefit to having such functions already implemented in libraries that were known to be both optimised and peer-tested. 

Secondly, following research into the Python language, it soon became clear that it was a popular choice amongst scientific and engineering teams due mainly to its impressive level of support for scientific and numeric computing \cite{perez} and promotion of `natural language' within its syntax that is known to aid in supporting rapid prototyping \cite{ramanujam}. These were both characteristics that could be applied to the needs of this project, and hence were deemed beneficial to future development.

%Thirdly, the process and technologies used to deploy Python applications appeared to be significantly less complicated than that for C++ applications. While  

A further reason for switching to the Python language, which would in time become one of the most significant, was the prospect of being able to make use of the iPython interactive computing library \cite{ipython}. Following the discovery of this library during research looking into the use of Python within scientific applications, it became apparent that it had the potential to provide significant capabilities in terms of supporting the automated execution and presentation of experiments and their results. This approach provided a more  intuitive means of displaying and analysis than what was originally available using just a command-line based interface. 

One of the most commonly noted disadvantages of using Python describes the fact that as a scripted language, its general performance is considered to be slower overall than that of compiled languages such as C++. While this was initially of some concern, it was later addressed through the use of a third-party library (Cython \cite{cython}) that enabled computationally expensive functions originally written in Python to be converted automatically into compiled C code.

\subsection{Development Tools}
\label{develtools}

Given the author's lack of prior knowledge in C++ or Python, it was decided that the use of an IDE for development would be most suitable for development, as they would be able to provide support facilities including code completion, automatic refactoring and powerful debugging facilities. 

For C++ development, the Xcode IDE \cite{xcode} was chosen. While this was primarily a decision based on ease of having the IDE already installed, Xcode did provide a high level of support for managing the deployment of C++ applications, including a facility to automatically generate the `Makefiles' used to install the released version on another computer. 

Python development was conducted within the PyCharm IDE making use of the educational license provided by JetBrains. Of particular benefit was the support provided for the automatic management of external dependencies and libraries, which included the ability to export a complete list of a project's dependencies and their precise version numbers.

Version control was managed by git, with repositories hosted on both Github and a separate private git server. The project also made use of the Travis CI continuous integration platform to provide automated building and testing of software. Integrating directly with the Github service, new builds would be triggered automatically upon pushing of a new commit to the remote repository. Should any part of the build process fail (including unit test failures), the service would send an email notification detailing the type and source of the error.  

\subsection{Core Libraries}

While making use of only a small selection of libraries, these libraries were used rather extensively, in particular opencv and bumpy. python and python were pretty big aswell.

The project made use of a selection of libraries to support in the development of the both the experiment testing framework, and the experiment methods themselves.

The largest single contributing library was \textit{OpenCV} \cite{opencv}. Originally developed by Intel, \textit{OpenCV} is one of the most popular open source computer vision and machine learning libraries currently available, providing over 2500 optimised functions that implement some of the most renowned computer vision algorithms including SIFT (Scale Invariant Feature Transform)\footnote{\url{http://docs.opencv.org/modules/nonfree/doc/feature_detection.html}}, Viola-Jones face detection\footnote{\url{http://docs.opencv.org/doc/tutorials/objdetect/cascade_classifier/cascade_classifier.html}} and Lucas-Kanade optical flow\footnote{\url{http://docs.opencv.org/modules/video/doc/motion_analysis_and_object_tracking.html#calcopticalflowpyrlk}}. 

\textit{OpenCV} became a vital tool in the development of this project, providing a large number of ``core" image processing and computer vision facilities including colour space manipulation, sub-region extraction and template matching (see Section \ref{templmatchopencv}). This project referenced the latest stable release at the time of writing (2.11.0), and made use of both the C++ and Python bindings as discussed in Section \ref{develtools}.

For aspects of the implementation developed in Python (approximately 75\% of code written), the project made extensive use of the \textit{Numpy} \cite{numpy}, \textit{Cython} \cite{cython}, \textit{Matplotlib} \cite{matplotlib} and \textit{iPython} \cite{ipython} libraries. 

\textit{Numpy} \cite{numpy} is a very popular third-party extension library providing optimised array-type objects in addition to a wide range of mathematical operations associated with numeric collections including matrix manipulation\footnote{\url{http://docs.scipy.org/doc/numpy/reference/routines.linalg.htm}}, statistical analysis\footnote{\url{http://docs.scipy.org/doc/numpy/reference/routines.statistics.html}} and Fourier transforms\footnote{\url{http://docs.scipy.org/doc/numpy/reference/routines.fft.html}}. One of the core data structures provided by \textit{Numpy} is the multidimensional array\footnote{\url{http://docs.scipy.org/doc/numpy/reference/generated/numpy.ndarray.html}}, which happened to be the data structure of choice within the \textit{OpenCV} Python bindings for representing an image.

The \textit{Cython} library \cite{cython} is an advanced tool utilised within the project for translating computationally expensive functions written originally in Python, into optimised C code. The library came with its own dialect of Python, that while being almost identical, enabled developers to add additional `C-specific instructions' (such as specifying type declarations for variables) to allow for full conversion from Python to C without the need for the developer to write any C code themselves. \textit{Cython} source files were distinguished from ``true" Python source files through the use of the \textit{``.pyx"} file extension (as opposed to the \textit{``.py"} extension for Python source files). While \textit{Cython} did also possess the ability to convert native Python code, the resulting compiled C files would not be fully optimised.

\textit{Matplotlib} \cite{matplotlib} is a popular plotting library for Python providing high-level support for producing a wide range of scientific graphs and figures in both 2D and 3D. It was used extensively throughout this project for depicting experiment results and analysing method behaviour (e.g. plotting the behaviour of template matching similarity scores).

\textit{iPython} \cite{ipython} provides an interactive environment for running computational operations where a dedicated graphical user interface may not be appropriate. It is becoming very popular amongst the scientific and data-analysis communities as a means of quickly and easily implementing, running and sharing computational tasks and experiments within an intuitive web-based environment (known as an \textit{iPython notebook}\footnote{http://ipython.org/notebook.html}).
 
\subsection{Implementation of Template Matching Similarity Measures}
\label{templmatchopencv}

All of the experiments conducted within this project focussed on the analysis of four specific template matching similarity measures for identifying corresponding sub-regions between two subsequent images of a forward-motion sequence.

The general process for performing template matching, regardless of the choice of similarity metric, remains fundamentally the same \cite{opencvtemplatematching}:

\begin{algorithm}
\caption{Template Matching}
\label{templatematchalgorithm}
\begin{algorithmic}[1]

\Procedure{Template\_Matching}{\textit{template\_image}, \textit{search\_image}}

\State let \textit{high\_score} = $-1$
\State let \textit{high\_score\_position} = $(-1, -1)$ \Comment{Initialise high score and position.}
\State let \textit{template\_image\_height} = len(\textit{template\_image})
\State let \textit{template\_image\_width} = len(\textit{template\_image}[0])\\\\

\Comment{Convolve the template image through the image we are searching.}
\For{$i \coloneqq 0$ \textbf{to} (len(\textit{search\_image}) - 1) - \textit{template\_image\_height} \textbf{step} $1$}
\For{$j \coloneqq 0$ \textbf{to} (len(\textit{search\_image[0]}) - 1) - \textit{template\_image\_width} \textbf{step} $1$} \\

\State let current\_window = search\_image[$i$][$j$]
\State let current\_match\_score = CHECK\_SIMILARITY(\textit{template\_image}, \textit{current\_window}) \\

\If{(\textit{high\_score} == $-1$) \textbf{or} (\textit{current\_match\_score} \textbf{is} better than \textit{high\_score})}

\State \textit{high\_score} = \textit{current\_match\_score}
\State \textit{high\_score\_position} = $(i, j)$

\EndIf \\

\EndFor
\EndFor \\

\\ \Return \textit{high\_score\_position} \\
\EndProcedure

\end{algorithmic}
\end{algorithm}

where a template image (i.e. the image we are searching for) is convolved through a (typically) larger search image (i.e. moved through the search image 1 pixel at a time). At each increment in position,  a similarity ``score" is calculated using a specific similarity metric calculation, if this new score is deemed ``better" than the current high score (which can be lower or higher depending on the similarity metric chosen) then that position within the search image becomes the new location where the item in the template image is most likely to be.

The similarity measures subsequently implemented in the experiments were chosen specifically in order to compare the performance between the two major categories of appearance-based matching technique. 

\subsubsection{Distance-Based Similarity Measures}

The first two measures were implementations of \textit{distance-based} matching. Under this approach (also known as \textit{correlation-based} matching), the similarity between two images is calculated by comparing the pixel-wise properties of each image (i.e. each pixel is compared with the corresponding pixel in the other image) using a particular type of measure. \cite{szeliski}. 

For this investigation, the \textit{Euclidean Distance} and \textit{Normalised Cross-Correlation} measures were selected as they both demonstrated differences that were suitable for comparison with respect to their approach to calculating pixel-similarity and robustness to changes in lighting. 

In the field of mathematics, the Euclidean Distance is a measure of the length of the vector that connects two distinct points located within Euclidean space \cite{szeliski}. In the context of appearance-based template matching, it is possible to utilise the Euclidean Distance to calculate a measure of similarity between pixel values of two images as defined by:

\begin{equation}
d(x, y) = \sqrt{\sum\limits_{i}\sum\limits_{j}[\mathcal{I}(x + i, y + j) - \mathcal{A}(i, j)]^2}
\end{equation}

where $\mathcal{A}$ is the template image containing the pattern we are searching for within image $\mathcal{I}$. 

A crucial characteristic of the Euclidean Distance (and indeed similar approaches including Sum of Squared Difference), is that it relies on comparing the \textit{intensity} of pixels in order to provide a measure of image similarity. This is very important, as it subsequently enforces an implicit assumption (known as the \textit{brightness constancy constraint}), that the values of corresponding pixels do not change between the two images\cite{szeliski}. This in effect, means that the measure can be substantially affected by changes in brightness between the two images under comparison.  

Normalised cross-correlation takes its roots from the field of signal processing, and unlike Euclidean Distance does not rely on differences in pixel intensity to establish the similarity between two images \cite{szeliski}. Instead, this approach takes the product (known as the \textit{cross-correlation}) of pixel values as defined by:

\begin{equation}
d(x, y) = \sum\limits_{i}\sum\limits_{j}[\mathcal{I}(x + i, y + j) \cdot \mathcal{A}(i, j)]^2
\end{equation}

However as it stands, this equation currently remains susceptible to the effects of differences in image brightness just in the same way as the Euclidean Distance. It is however possible to overcome this issue by first \textit{normalising} the pixel values prior to scoring: 

\begin{equation}
d(x, y) = \frac{\sum\limits_{i}\sum\limits_{j}[\mathcal{I}(x + i, y + j) \cdot \mathcal{A}(i, j)]^2}{\sqrt{\sum\limits_{i}\sum\limits_{j}[\mathcal{A}(i, j)]^2 \cdot \sum\limits_{i}\sum\limits_{j}[\mathcal{I}(x + i, y + j)]^2}}
\end{equation}

As well as dramatically improving resilience to differences in image exposure, normalising also guarantees the returned score will fall between -1 and 1 \cite{szeliski}. This can prove helpful when used as part of larger application. 

Given that both Euclidean Distance and Normalised Cross-Correlation were already available as optimised functions within the \textit{OpenCV} library (\texttt{norm} (specifically the \texttt{L2\_NORM} type)\footnote{\url{http://docs.opencv.org/modules/core/doc/operations_on_arrays.html#norm}} and \texttt{CV\_TM\_CCORR\_NORMED}\footnote{\url{http://docs.opencv.org/doc/tutorials/imgproc/histograms/template_matching/template_matching.html}} respectively), it was decided that in terms of performance, no real benefits would be gained from choosing to re-implement the functions manually. However during early stages of the investigation, some attempts were made to implement these functions regardless as part of a drive to enhance the author's own understand of these key topics.

By choosing to keep within the OpenCV eco-system, it also became possible to utilise the extensive template-matching facilities already supported within the library through the use of the \texttt{matchTemplate}\footnote{\url{http://docs.opencv.org/modules/imgproc/doc/object_detection.html?highlight=matchtemplate#matchtemplate}} function.

%
%As the image patches that are compared share the \textit{same dimensions}, the final distance function becomes a modified version of the \textbf{Sum of Squared Differences}:
%
%\begin{equation}
%d(x, y) = \sum\limits_{i}\sum\limits_{j}[\mathcal{I}(x, y) - \mathcal{A}(i, j)]^2
%\label{eq:ed2}
%\end{equation}

\subsubsection{Histogram-Based Similarity Measures}

An alternative approach to calculating a measure of similarity of two images using only their ``appearance", is that of \textit{histogram-based} matching.

While also focussing on the value of image pixels as a measure of similarity, this approach differs from distance-based approaches by choosing to represent and later compare this data as \textit{histograms}, rather than comparing between the actual value of corresponding pixels directly.

By choosing to represent an image as a histogram, it becomes possible to organise the various data aspects that of an image's appearance into a collection of value-specific counts (known otherwise as \textit{bins}). One of the advantages to using histograms is that it becomes easy to generate alternative representations of the same image, where each focusses upon a specific type of `metric' for describing that image (e.g. A separate histogram for each channel in the `RGB' colour space). 

In order to use histogram representations for the comparison of two or more images, the histogram for each image must first be calculated. As such, when performing histogram-based matching a couple of additional stages are required:

\begin{enumerate}
	\item Compute the histogram for the first image;
	\item Compute the histogram for the second image;
	\item Perform comparison between the first and second histograms;
\end{enumerate}

This is in contrast to distance-based approaches, which as a result of working within the original representation of an image, can be completed within a single comparison operation. 

The approach to comparing between two histograms follows closely with the method of comparison adopted within distance-based matching, whereby a specific \textit{distance metric} is calculated in order to provide a measure of similarity \cite{bradski2008learning}. 

The two metrics selected for evaluation within this investigation were \textit{Correlation} and \textit{Chi-Square}. Unlike the distance-based metrics chosen previously, these were selected entirely at random from four possible similarity measures provided by the \textit{OpenCV} library (Correlation, Chi-Square, Intersection and Bhattacharyya distance)\footnote{\url{http://docs.opencv.org/doc/tutorials/imgproc/histograms/histogram_comparison/histogram_comparison.html}}.

The Correlation metric is defined within the \textit{OpenCV} library \cite{opencvhistogram} as:

\begin{equation}
d(H_{1}, H_{2}) = \frac{\sum\nolimits_{I}[(H_{1}(I) - \bar{H_{1}})(H_{2}(I) - \bar{H_{2}})]}{\sqrt{\sum\nolimits_{I}[(H_{1}(I) - \bar{H_{1}})^2(H_{2}(I) - \bar{H_{2}})^2]}}
\end{equation}

where

\begin{equation}
\bar{H_{n}} = \frac{1}{N}\sum\limits_{J}[H_{n}(J)]
\end{equation}

and $N$ represents the number of `bins' used to organise histogram data. 

When scoring the similarity between two histograms via correlation, a higher score represents a better match, with 1 representing a complete match, 0 representing that no correlation could be obtained (i.e. random) and -1 representing a complete mismatch \cite{bradski2008learning}.

Originating from the field of statistical analysis, the Chi-Square metric defines a class of tests that can be used to determine how closely a collection of `observed' data matches the data that would be expected according to a particular hypothesis \cite{mclaughlin}. 

For the purposes of comparing image histograms, the \textit{OpenCV} library provides an implementation of a specific type of chi-squared test known as Pearson's chi-square test \cite{bradski2008learning}:

\begin{equation}
d(H_{1}, H_{2}) = \sum\limits_{I}\frac{(H_{1}(I) - H_{2}(I))^2}{H_{1}(I) + H_{2}(I)}
\end{equation}

For chi-square, scoring is conducted in the opposite fashion to correlation, where a lower score indicates a better match. In this case, 0 represents a perfect match and any value greater than 0 represents a worse match.

Just as in the case of the distance-based metrics, the \textit{OpenCV} library provided optimised support for computing and subsequently comparing image histograms via the \texttt{calcHist}\footnote{\url{http://docs.opencv.org/modules/imgproc/doc/histograms.html#calchist}} and \texttt{compareHist}\footnote{\url{http://docs.opencv.org/modules/imgproc/doc/histograms.html#comparehist}} functions respectively. 

While histograms provide an alternative representation of an image, they still contain fundamentally the same data. This means that in the case where one image is significantly brighter than the other, it becomes unlikely that a histogram-based technique will be successful in identifying a correct match or non-match. 

Using the same underlying theory as Normalised Cross-Correlation, it was possible to help minimise brightness-relating matching errors by ensuring that both histograms were normalised. Although none of the histogram similarity metrics included a built-in normalisation step (as is the case with Normalised Cross-Correlation), \textit{OpenCV} provided a specific \texttt{normalize}\footnote{\url{http://docs.opencv.org/modules/core/doc/operations_on_arrays.html#normalize}} function that enabled the histograms to become normalised prior to any comparison.

\subsubsection{Colour Space}

While the normalisation of image data can help reduce matching errors caused by changes in brightness between two images, a means of further enhancing this resilience is to convert the colour space of both images to one that is capable of separating the \textit{colour} of a pixel from its \textit{brightness} \cite{ulrich-nourbakhsh}. 

As a result it then becomes possible to compare the two images using only the \textit{colour information}, effectively ignoring the effects of changing brightness that can cause two pixels that are in fact visually the same, to be deemed as different.

Within this investigation, all testing images were converted from RGB (loaded as BGR in \textit{OpenCV}) to the \textit{Hue-Saturation-Value} colour space prior to the beginning of any experiment tests. This was achieved using the \texttt{cvtColor}\footnote{\url{http://docs.opencv.org/modules/imgproc/doc/miscellaneous_transformations.html#cvtcolor}} \textit{OpenCV} function with the \texttt{BGR2HSV} conversion type parameter. This later allowed the `Value' channel to be ignored as part of all image comparisons made during the template matching procedures conducted within an experiment.

\subsection{Execution of Experiment Methods}

\subsubsection{Automated Test Rig}

One of the major design aspects of the project was the decision to implement an automated testing rig to facilitate in the conducting of experiments. The core motivation behind the test rig, was to find an efficient approach to executing experiments wishing to test a wide combination of potential method parameters, in a manner that was both comprehensive and repeatable. 

While of course defining these parameters would remain the responsibility of the user, the primary use case of the system saw the user specify all of the possible test parameters \textit{up front}, before instructing the test rig to repeatedly perform and document each experiment in isolation, until all possible combinations of the test parameters for that experiment had been attempted. The most obvious advantage to this system was the significant reduction that could be made in terms of human resources. Thus allowing the user to focus on tasks that were less suited or perhaps not possible for a computer to achieve, such identifying key aspects within the results gathered from the last run of the test rig. 

However, another advantage perhaps not as immediately visible, was the almost complete removal of the risk for human error that otherwise (assuming a moderate number of potential parameters) would be set fairly high, given the situation where a user had to manually run each possible combination of experiment. Although, it should be noted that as the user would continue to interact in some form, there was always still a risk of experiment results being tainted as a consequence of human error being introduced. 

% Add UML USECASE DIAG

In terms of interaction with the user, the test rig provided a command-line based interface that provided support for the input of all test parameters and configuration options using a collection of specific parameter arguments. While certain arguments related to specific experiment, a number of arguments common to all experiments conducted under this investigation were defined as follows:

\begin{table}[h]
\small
\begin{tabular}{|p{2cm}|p{2cm}|p{2cm}|p{4cm}|p{3cm}|}
\hline
Parameter Name & Argument Specifier & Type  & Input Format & Description \\ \hline
Image Pairs & \texttt{--images} & \texttt{(String, String)[]} & \texttt{"\textless pair\_1\_image\_1\textgreater, \textless pair\_1\_image\_2\textgreater", "\textless pair\_1\_image\_1\textgreater, \textless pair\_1\_image\_2\textgreater"} & test \\
\hline   
\end{tabular}
\end{table}



\subsubsection{Library of `Core' Functions}

\subsection{Display \& Analysis of Results}

\subsection{Testing?}

- Unit test suites
- Coveralls.io

\section{Experiment Methods}

\subsection{Experiment 1: Template Matching (Multiple Small Patches)}
\label{ex1}
\subsection{Experiment 2: Template Matching (Full-width Patches - Non-Scaled)}
\subsection{Experiment 3: Template Matching (Full-width Patches - Geometric Scaled)}
\label{tm-scaled}

\subsection{Additional Experiments}

\subsection{Feature-based matching}
\subsection{Edge-based matching}




