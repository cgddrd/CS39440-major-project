\thispagestyle{empty}

\begin{center}
    {\LARGE\bf Abstract}
\end{center}


The ability for an autonomous robot to navigate from one location to another in a manner that is both safe, yet objective is vital to the survivability of the machine, and the success of the mission it is undertaking. 

%Over recent years, a great amount of research work has focussed on the use of visual stimuli as a means of obtaining information about the current condition of the local terrain. 

While the problem of vision-based obstacle detection has enjoyed a high level of research focus over recent years, it would appear that much less work has concentrated on providing a means of reliably detecting changes in terrain slope through the exploitation of observed changes in motion.

Making use of recent advances in camera technology with appropriate computer vision techniques, this project focussed on conducting an investigation into the use of appearance-based template matching technique in conjunction with optical flow analysis to provide an estimation into the current gradient conditions of the environment terrain, while also supporting the detection of nearby obstacles, and providing an indication as to the egomotion properties of a mobile robot.

While not all of the original research aims were met, progress was made in terms of establishing a positive relationship between the vertical position of a feature and the level of vertical displacement that it demonstrates between two subsequent images. The investigation saw three key experiments conducted, all focussing on improving the approach to using appearance-based template matching with the aim of accurately determining the level of vertical displacement shown, and thus subsequently allowing the detection of potential changes in slope, or appearance of obstacles within the current field of view.

Although the final results were generally mixed, there were occasions where key conclusions could be drawn as to the performance characteristics of each evaluated experiment method. However due to the overall lack of consistency in the results, this project has laid the groundwork to contribute towards further development within this research area.


