\documentclass[10pt, compress]{beamer}

\usetheme{m}

\usepackage{booktabs}
\usepackage[scale=2]{ccicons}
\usepackage{minted}
\usepackage{wrapfig}
\usepackage{enumitem}% http://ctan.org/pkg/enumitem

\usepgfplotslibrary{dateplot}

\usemintedstyle{trac}

\title{Estimation of Terrain Gradient Conditions \& Obstacle Detection Using a Monocular Vision-based System}
\subtitle{Final Demonstration}
\date{\today}
\author{Connor Luke Goddard (clg11)}
\institute{Department of Computer Science, Aberystwyth University}

\begin{document}

\maketitle

\plain{Background}

\begin{frame}[fragile]
  \frametitle{Background}
	
     \begin{block}{}  

	          \begin{wrapfigure}{r}{0.4\textwidth}
   \vspace{-20pt}
  \begin{center}
    \includegraphics[width=0.25\textwidth]{mars.jpg}
  \end{center}
  \vspace{-15pt}
  \caption{\small Tracked features from MER \cite{mars}.}
  \end{wrapfigure}
  
  The ability for an autonomous robot to navigate from one location to another in a manner that is both \textbf{safe}, yet \textbf{objective} is vital to the \textbf{survivability} of the machine, and the \textbf{success of the mission} it is undertaking.    
		
       \end{block}

\vspace{15pt}
	
   \begin{block}{}
Vision based obstacle detection has enjoyed a high level of research focus over recent years. Less work has concentrated on providing a means of reliably detecting changes in terrain slope through the exploitation of observed changes in motion. 
  \end{block}


\end{frame}

\begin{frame}[fragile]
  \frametitle{Related Work}
  
  	\begin{enumerate}[label={[\arabic*]}]
  	
  		\item \textbf{\textit{A Robust Visual Odometry and Precipice Detection
System Using Consumer-grade Monocular Vision}}, J. Campbell \textit{et al.}, IEEE, 2005

		\item \textbf{\textit{Obstacle Detection using Optical Flow}}, T. Low and G. Wyeth, School of Information Technology and Electrical Engineering, University of Queensland, 2011
		
		\item \textbf{\textit{Appearance-based Obstacle Detection with Monocular Color Vision}}, I. Ulrich and I. Nourbakhsh, AAAI, 2000
  			
  	\end{enumerate}
  	
  	\vspace{5pt}
  	
  	
  	\begin{figure}[ht!]
\centering
\includegraphics[scale=0.3]{related_work.png}
    \caption{\textbf{Left:} Precipice detection [1], \textbf{Right:} Obstacle colour detection [3]}
  \end{figure}

  	
\end{frame}

\begin{frame}[fragile]
  \frametitle{Project Overview}
  
  \begin{block}{Focus}
    Investigation into a system capable of utilising a single, forward-facing colour camera to provide an estimation into current terrain gradient conditions, obstacle location \& characteristics and robot ego-motion. 
  \end{block}
  
  
  \begin{block}{Potential Metrics For Observation}
   
  \begin{itemize}[label={\textbullet}]
  	\item Changes in terrain gradient (i.e. slopes).
  	\item Location \& characteristics of positive and negative obstacles (i.e. rock or pit respectively).
  	\item Current robot speed of travel.
  	\item Changes in robot orientation.
  \end{itemize}
  
 \end{block}
 
\end{frame}

\plain{Working Hypothesis}

\begin{frame}[fragile]
  \frametitle{Working Hypothesis}

	\hspace*{20pt} \textit{``From images captured using a \textbf{non-calibrated} monocular camera system, analysis of optical flow vectors extracted using \textbf{localised appearance-based} dense optical flow techniques can provide certain estimates into the current condition of terrain gradient, the location and characteristics of obstacles, and robot ego-motion.''}

\end{frame}

\begin{frame}[fragile]
  \frametitle{Motion Parallax}

  Approach focussed on the effects of \textbf{motion parallax}. \\ \vspace{0.5cm}
  
  \textbf{i.e.} Objects/features that are at a greater distance from the camera appear to  move less from frame-to-frame than those that are closer.
  
\begin{figure}[ht!]
\centering
\includegraphics[scale=0.2]{motion_parallax.png}
    \caption{Typical example of motion parallax. \textbf{Courtesy}: \href{http://www.infovis.net}{http://www.infovis.net}.}
  \end{figure}
  
\end{frame}

\begin{frame}[fragile]
  \frametitle{Inference of Terrain Gradient \& Obstacle Detection}

  \begin{block}{Key Aspects}
   
  \begin{enumerate}[label={\arabic*.}]
  \item The exclusive use of appearance-based template matching techniques to provide a localised variant of dense optical flow analysis over the use of sparse optical flow techniques relying upon a high number of available image features.
  \item The use of a formalised model to represent detected changes in vertical displacement as part of efforts to estimate changes in terrain gradient in addition to the location characteristics of potential obstacles. 
\end{enumerate}
  
 \end{block}
 
\end{frame}

\begin{frame}[fragile]
  \frametitle{Vertical Displacement Model}
  
  \vspace{-20pt}

 Provides a mechanism by which to evaluate the prediction:
 
 \vspace{10pt}
 
       \begin{wrapfigure}{r}{0.4\textwidth}
   \vspace{-30pt}
  \begin{center}
    \includegraphics[width=0.42\textwidth]{model.pdf}
  \end{center}
  \vspace{-15pt}
  \caption{\small Tracked features from MER \cite{mars}.}
  \end{wrapfigure}
   
%   \hspace*{20pt} 
   
  \textit{``Features captured towards the \textbf{bottom} of an image will show \textbf{greater displacement} between subsequent frames, than those features captured towards the \textbf{top} of the image."}
   
\end{frame}

\begin{frame}[fragile]
  \frametitle{Vertical Displacement Model}

From observing discrepancies between the current model and ``baseline" model, it should be possible to infer the presence of potential obstacles: 

  \begin{figure}[ht!]
\centering
\includegraphics[scale=0.4]{obstacle_graph}
  \end{figure}
  
  
\end{frame}

\begin{frame}[fragile]
  \frametitle{Vertical Displacement Model}

Differences in model discrepancy behaviour should provide enough information to establish between a potential obstacle and a change in terrain slope: 

  \begin{figure}[ht!]
\centering
\includegraphics[scale=0.32]{slope_vs_obstacle}
  \end{figure}
  
  
\end{frame}

\begin{frame}[fragile]
  \frametitle{Investigation Aims}

  
\end{frame}

\plain{Experiment Methods}

\begin{frame}[fragile]
  \frametitle{Experiment Methods}
  
\end{frame}

\begin{frame}[fragile]
  \frametitle{Experiment 1}
  
\end{frame}

\begin{frame}[fragile]
  \frametitle{Experiment 2}
  
\end{frame}

\begin{frame}[fragile]
  \frametitle{Experiment 3}
  
\end{frame}

\begin{frame}{Current Approach}

%Input: Two images with a set displacement between them (e.g. 10cm). \\

\textbf{Stage One} \\ \vspace{0.2cm}

Import two consecutive images, and convert from RGB (BGR in OpenCV) colour space to HSV.

\begin{figure}[ht!]
\centering
\includegraphics[scale=0.26]{rgb2hsv.png}
  \end{figure}
  
The 'V' channel is then removed in order to improve robustness to lighting changes between frames. 
\end{frame}

\begin{frame}{Current Approach}

\textbf{Stage Two} \\ \vspace{0.2cm}

Extract a percentage-width region of interest (ROI) from centre of first image.

\begin{figure}[ht!]
\centering
\includegraphics[scale=0.5]{stage1.png}
  \end{figure}
  
\end{frame}

\begin{frame}{Region-of-Interest}

\textbf{Why do we need to extract a ROI?} \\ \vspace{0.5cm}

\textbf{Focus-of-expansion}: Objects in images do not actually move in 1-dimension (i.e. straight down the image). \\ \vspace{0.2cm}

This effect is minimised towards the centre of the image.

\begin{figure}[ht!]
\centering
\includegraphics[scale=0.15]{foe.png}
\caption{\textbf{Courtesy:} J.Pillow, University of Texas}
  \end{figure}
  
\end{frame}

\begin{frame}{Current Approach}

\textbf{Stage Three} \\ \vspace{0.2cm}

Extract patches of a fixed size around each pixel within the extracted ROI.

\begin{figure}[ht!]
\centering
\includegraphics[scale=0.4]{stage2.png}
\caption{\textbf{Simplified} example of patch extraction within ROI.}
  \end{figure}
  
\end{frame}

\begin{frame}{Current Approach}

\textbf{Stage Four} \\ \vspace{0.2cm}

For each patch extracted from \emph{image one}, move down through a localised search window (column) in \emph{image two} searching for the best match against the template patch. 

\begin{figure}[ht!]
\centering
\includegraphics[scale=0.35]{stage3.png}
\caption{Example of ``best match" search within local column.}
\end{figure}
  
\end{frame}

\begin{frame}{Current Approach}

\textbf{Stage Five} \\ \vspace{0.2cm}

Identify the patch within the localised search window that provides the ``best match" via correlation-based matching (e.g. Euclidean Distance, SSD or Correlation coefficient). 

\begin{figure}[ht!]
\centering
\includegraphics[scale=0.35]{stage4.png}
\end{figure}

\end{frame}

\begin{frame}{Current Approach}

\textbf{Stage Six} \\ \vspace{0.2cm}

Average all measured displacements for each pixel along a given row.

\begin{figure}[ht!]
\centering
\includegraphics[scale=0.4]{stage5.png}
\end{figure}

Outliers are removed by ignoring any displacements that lie outside of (2 x Standard Deviation) of the mean.

\end{frame}

\begin{frame}{Current Approach}

\textbf{Repeat Stages 1-6} \\ \vspace{0.2cm}

Repeat stages 1-6 for an entire collection of ``calibration" images taken of \emph{flat, unobstructed} terrain.

\begin{figure}[ht!]
\centering
\includegraphics[scale=0.3]{calibimages.png}
\end{figure}

\end{frame}

\begin{frame}{Current Approach}

\textbf{Stage Seven} \\ \vspace{0.2cm}

Plot the \emph{average displacement} for each ROI row, calculated from the displacements recorded over all calibration images.

\begin{figure}[ht!]
\centering
\includegraphics[scale=0.45]{graph.png}
\end{figure}

\end{frame}


\begin{frame}[fragile]
  \frametitle{Exhaustive vs. Non-Exhaustive}

  
\end{frame}

\plain{Investigation Results}

\begin{frame}{Results}

Generally ``mixed" results at this stage. \\ \vspace{0.5cm}

From the results obtained, we have learned:

\begin{enumerate}[label={\arabic*.}]
  \item It is possible to potentially establish a relationship between row height in an image, and average downwards pixel displacement.
  \item The current approach to appearance-based tracking using multiple ``small" patches\textbf{does not work well} for many typical terrain conditions (i.e. outside). 
\end{enumerate}

\end{frame}

\begin{frame}{Results: Example 1}

\begin{columns}[T] % align columns
\begin{column}{.48\textwidth}

\textbf{Input Collection:}

\begin{figure}[ht!]
\centering
\vspace{0.3cm}
\includegraphics[scale=0.18]{calibimages_good.png}
\end{figure}

\end{column}%
\hfill%
\begin{column}{.48\textwidth}

\textbf{Result:}

\begin{figure}[ht!]
\centering
\includegraphics[scale=0.22]{flat_10cm_results.png}
\end{figure}

\end{column}%
\end{columns}

\end{frame}

\begin{frame}{Results: Example 2}

\begin{columns}[T] % align columns
\begin{column}{.48\textwidth}

\textbf{Input Collection:}

\begin{figure}[ht!]
\centering
\vspace{0.3cm}
\includegraphics[scale=0.18]{calibimages.png}
\end{figure}

\end{column}%
\hfill%
\begin{column}{.48\textwidth}

\textbf{Result:}
\begin{figure}[ht!]
\centering
\includegraphics[scale=0.22]{wiltshire_outside_10cm.png}
\end{figure}

\end{column}%
\end{columns}

\end{frame}


\plain{Evaluation \& Conclusion}


\begin{frame}[fragile]
  \frametitle{Future Work}

  
\end{frame}

\begin{frame}[fragile]
  \frametitle{Future Work}

  
\end{frame}

\begin{frame}[fragile]
  \frametitle{Summary}

  
\end{frame}


%\begin{frame}{Project Management}
%
%Project is adopting a SCRUM-based approach to project management. \\ \vspace{0.2cm}
%
%\begin{itemize}
%	\item \textbf{Releases} - Major feature sets (e.g. work up to mid-project demo)
%	\item \textbf{Sprints} - Weekly time-boxed portion of work effort.
%	\item \textbf{Sprint Review \& Retrospectives} - Conducted as part of weekly project meeting with tutor. 
%\end{itemize}
%
%
%\textbf{Tools:} \textit{Github Issues} with \textit{\href{http://waffle.io}{Waffle.io}} (KANBAN), \textit{\href{http://burndown.io/}{Burndown.io}} (Burndown charts).
%	
%\end{frame}
%
%\begin{frame}{Project Management}
%
%\begin{figure}[ht!]
%\centering
%\includegraphics[scale=0.12]{waffle.png}
%\caption{Waffle.io KANBAN board interface to Github Issues.}
%\end{figure}
%
%\begin{figure}[ht!]
%\centering
%\includegraphics[scale=0.12]{burndown.png}
%\caption{Burndown chart for Release v1.0. \textbf{Courtesy:} \href{http://burndown.io}{http://burndown.io}}
%\end{figure}
%
%\end{frame}
%
%\begin{frame}{Next Steps}
%
%Plan is to move away from multiple small patches instead adopting a single, larger patch to represent a single row. \\ \vspace{0.5cm}
%
%Template patch will be scaled relative to its centre as the search moves down the image. This is to account for perspective distortion (\textbf{i.e.} objects becoming larger as they move closer to the camera).
%
%\begin{figure}[ht!]
%\centering
%\includegraphics[scale=0.4]{scaling.png}
%\end{figure}
%	
%\end{frame}

\plain{Questions? \\ \vspace{0.2cm} \footnotesize{\href{http://mmpblog.connorlukegoddard.com}{Project Blog} \\ \vspace{0.2cm} \footnotesize{Slide Design: Matthias Vogelgesang - (\href{https://github.com/matze/mtheme}{Github})}}}

\end{document}
